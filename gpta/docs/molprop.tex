\subsubsection{Molecular Properties}
\index[action]{Molecular Properties}
\index[command]{--molprop}

This action computes a variety of molecular properties of the system.
The properties that can currently be computed are
\begin{itemize}
\item The molecular dipole. \\
	In order to compute this property the atomic charges must be present in the input files or defined using the \emph{--define} action.
	Regardless of the method, the atomic charges are set in the first frame and kept unchanged throughout the entire trajectory.
	This is because not all coordinates formats contain the coordinates, but it could be easily changed by commenting a line in the \verb|readCoordinatesModule.F90| file, around line 335.
\item The polar angle of the molecular dipole.\\
	See the previous property for a discussion about the atomic charges.
\item The polar angle for the vector normal to the plane defined by three atoms in the molecule.
\item The angle between three atoms in the molecule.
\item The torsional angle between four atoms in the molecule.
\item The radius of gyration.\\
\[
  R_\mathrm{g}^2 \ \stackrel{\mathrm{def}}{=}\ 
  \frac{1}{M} \sum_{k=1}^{N} m_k\left( \mathbf{r}_k - \mathbf{r}_0 \right)^2
\]
where $ \mathbf{r}_0$ is the position of the centre of mass, $\mathbf{r}_k$ and $m_k$ are the position and mass of atom $k$, and $M$ is the total mass of the $N$ selected atoms.
\end{itemize}

The properties can then dumped to a file (\emph{+dump}) or internally processed to compute
\begin{itemize}
\item their average:  \emph{+avg}
\item their distribution: \emph{+dist}
\item their average in slices normal to the $z$ direction: \emph{+distZ}
\item thier 2D distribution with the $z$ coordinate of the molecule's centre of mass: \emph{+dist2D}
\end{itemize}

\hdashrule[0.5ex][x]{\textwidth}{1pt}{3mm}

\begin{itemize}
\item[Command:] \emph{--molprop}

\vspace{0.5cm}
\item[Prerequisite:] \emph{--top}

\vspace{0.5cm}
\item[List of flags:] 
\emph{+out [molprop.out]}\\
  defines the output file name

\emph{+id [molecule name]}\\
  selects the molecule(s) to be used in the calculation.

\emph{+normalZ [2,1,3]}\\
  computes the cosine of the polar angle of the vector normal to the plane formed by the three atoms specified.

\emph{+dipole}\\
  computes the molecules' dipole moment, requires the charges to be defined.

\emph{+dipoleZ}\\
  computes the cosine of the polar angle of the molecules' dipole.

\emph{+bond [1,2]}\\
  computes the bond distance between the three atoms specified. 	
  The first atom in the list is the central atom.

\emph{+angle [2,1,3]}\\
  computes the angle between the three atoms specified. 	
  The first atom in the list is the central atom.

\emph{+torsion [4,1,5,8]}\\
  computes the torsion angle for the four atoms specified.

\emph{+deg}\\
  transforms the angles from radians to degrees

\emph{+rgyr}\\
  computes the radius of gyration of the molecules.

\emph{+dumpSeq}\\
  dumps the computed property to the output file sequentially and two empty lines are added to separate the data of the individual frames. 
  This is the default option.

\emph{+dump}\\
  dumps the computed property to the output file, one line per frame and $N_{mol}$ columns.

\emph{+avg}\\
  computes the average of the property.

\emph{+dist [0,4]}\\
  computes the distribution of the property within the specified limits.
  For the distribution of angles it is often useful to set the limits in terms of $\pi$.
  This can be done by using the strings: \verb|pi|, \verb|-pi|, \verb|twopi|, \verb|-twopi|, \verb|pih|, \verb|-pih|, \verb|piq| and \verb|-piq|, which are interpreted as
  $\pi$, $-\pi$, $2\pi$, $-2\pi$, $\pi/2$, $-\pi/2$, $\pi/4$ and $-\pi/4$, respectively.

\emph{+distZ}\\
  computes the average property in slices along the $z$ axes.

\emph{+dist2D [0,4]}\\
  computes the distribution of property within the specified limits in slices along the $z$ axes.

\emph{+nbin [100]}\\
  defines number of bins to be used in the calculation of the property's distribution.

\emph{+nbinZ [100]}\\
  defines number of slices to be used in the $z$ direction.

\vspace{0.5cm}  
\item[Examples:]
\begin{small}
\begin{Verbatim}[frame=single]
gpta.x --i c.pdb --top --molprop +id M1 +angle 1,2,3 +avg
gpta.x --i c.pdb --top --molprop +id M1 +torsion 1,2,3,4 +dist -pi,pi +nbin 180
gpta.x --i c.pdb --top --molprop +id M1 +dipole +dist 0,4 +nbin 50 +out dipole.out
gpta.x --i c.pdb --top --molprop +id M1 +dipoleZ +distZ +nbinZ 100
gpta.x --i c.pdb --top --molprop +id M1 +normalZ 1,2,3 +dist2D 0,pi +nbinZ 100 +nbin 90
gpta.x --i c.pdb --top --molprop +id M1 +dipole +dump +out dipole.out
\end{Verbatim}
\end{small}

\vspace{0.5cm} 
\item[Screen output:]
\begin{scriptsize}
\begin{verbatim}
...
_________________________________________________________________________________________________________
Computing molecular properties
...Output file.....................................: molprop.out
...Molecule's type selected........................: M1
...Number of molecules selected....................:        3137
...Computing -> Covalent angle
......Atom indices used............................: 2 1 3
...Computing the property distribution
......Distribution limits..........................:     90.0000       110.0000
......Number of bins...............................:         100
_________________________________________________________________________________________________________
...
\end{verbatim}
\end{scriptsize}

%\vspace{0.5cm}
%\item[File output:]
%\begin{scriptsize}
%\begin{verbatim}
%...
%\end{verbatim}
%\end{scriptsize}

\end{itemize}

\hdashrule{\textwidth}{1pt}{}
