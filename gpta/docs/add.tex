\subsubsection{Add particles}
\index[action]{Add Particles}
\index[command]{--add}

This action adds new particles to a frame.
It can be used to add solute/solvent atoms or molecules, or to add dummy particles to the system.
The new species are added randomly to the cell.
This action is a wrapper for different actions which are called by the following commands,
which accept different flags and work in slightly different ways;
\begin{itemize}
\item --add solvent \\
This command adds a solvent to the input frame either as elements from the periodic table or as molecules read from a file.

\item --add solute \\
This command adds a solute to the input frame either as elements from the periodic table or as molecules read from a file.

\item --add centre \\
This command adds a dummy atom to a molecule, \emph{e.g.} the fourth site in TIP4P water.

\item --add dummy \\
This command adds the geometric centre (or the centre of mass) to a molecule as an extra particle, which can be used for visualisation purposes or for computing other properties related to the position of the centre of mass.

\item --add merge \\
This command combines the coordinated from two files.

\item --add box \\
This command creates a new system by defining its box. 
The box can be defined using 1 number (cubic), three numbers (orthorhombic), 6 numbers ($|a|, |b|, |c|, \alpha, \beta$ and $\gamma$) or 9 numbers (metric matrix $\vec{a},\vec{b},\vec{c}$).

\end{itemize}
In particular, for the first two commands, after the new particles have been added, the overlapping solvent molecules are removed.
Hence, in the case of \emph{--add solvent} the number of molecules added will be less than the number specified in the input, while in the case of \emph{--add solute} some of the molecules that were initially in the system will be removed.
No molecules will be removed after the other commands.

This action can also be used to create a system from scratch, \emph{i.e.} when no \emph{--i} command is used.


\hdashrule[0.5ex][x]{\textwidth}{1pt}{3mm}

\begin{itemize}
\item[Command:] \emph{--add solvent / -add solute}

\vspace{0.5cm}
\item[Prerequisite:] \emph{--top}

\vspace{0.5cm}
\item[List of flags:] 
\emph{+atom [species]}\\
  defines the name of mono-atomic species to add, it doesn't have to be an element from the periodic table..

\emph{+f [filename]}\\
  defines the name of the file that contains the coordinates of the molecule to add to the system.

\emph{+n [1]}\\
  sets number of species to add.

\emph{+rmin [1.5]}\\
  sets the cutoff distance to remove overlapping solvent molecules

\emph{+box [cell]}\\
  defines the size and shape of the region of space to be filled by the new molecules.
  If no cell was present, this will be the new global cell.
  See the reading coordinates action (\emph{--i}) for how the cell can be specified.

\emph{sphere [radius]}\\
  define the radius of the sphere to be filled by the new molecules.

\emph{+origin [0,0,0]}\\
  defines the origin of the box or the centre of the sphere to be filled by the new molecules.

\vspace{0.5cm}  
\item[Examples:]
\begin{small}
\begin{Verbatim}[frame=single]
gpta.x --i coord.pdb --top --add solute +atom Na,Cl +n 2,2 +rmin 4.0
gpta.x --i coord.pdb --top --add solvent +f benzene.pdb +n 5 +rmin 5.0 --o new.pdb
\end{Verbatim}
\end{small}

\hdashrule[0.5ex][x]{\textwidth}{1pt}{3mm}

\item[Command:] \emph{--add centre}

\vspace{0.5cm}
\item[Prerequisite:] \emph{--top}

\vspace{0.5cm}
\item[List of flags:]
\emph{+mol [molecule ID]}\\
  defines the name of the molecule whose centre of mass will be calculated.
  The molecules are seleceted using the names given to them by the \emph{--top} command.

\emph{+i i1,i2\dots} \\
  defines the indices of atoms in the molecules to be used for the calculation of the centre of mass.
  By default all atoms are used.

\vspace{0.5cm}
\item[Examples:]
\begin{small}
\begin{Verbatim}[frame=single]
gpta.x --i coord.pdb --top --add centre +mol M1
gpta.x --i benzene.pdb --top --add centre +mol M1 +i 1,3,5,7,9,11
\end{Verbatim}
\end{small}

\hdashrule[0.5ex][x]{\textwidth}{1pt}{3mm}

\item[Command:] \emph{--add dummy}

\vspace{0.5cm}
\item[Prerequisite:] \emph{--top}

\vspace{0.5cm}
\item[List of flags:]
\emph{+mol [molecule ID]}\\
  defines the name of the molecule where the dummy site will be added.
  The dummy particle is added along the bisector of the HOH bond.

\emph{+i i1,i2,i3} \\
  defines the indices of the O and H atoms that will be used to compute the bisector.
  The O atoms is the first index.
  
\emph{+dist [0.1]} \\
  defines the distance along the bisector where the dummy site will be added.

\vspace{0.5cm}
\item[Examples:]
\begin{small}
\begin{Verbatim}[frame=single]
gpta.x --i water.pdb --top --add dummy +i 1,2,3 +dist 0.1577 --o tip4p_ice.pdb
\end{Verbatim}
\end{small}


\item[Command:] \emph{--add box [box]}


\vspace{0.5cm}
\item[Examples:]
\begin{small}
\begin{Verbatim}[frame=single]
gpta.x --add box 12. --add solvent +f h2o.pdb +box 12. +n 57 +rmin 2 --top --o water.pdb
\end{Verbatim}
\end{small}


\item[Command:] \emph{--add merge}

\emph{+f} \\
  defines the file that contains the coordinates of the atoms to be added to the system.

\vspace{0.5cm}
\item[Examples:]
\begin{small}
\begin{Verbatim}[frame=single]
gpta.x --i coord.pdb --add merge +f kbr.pdb --top +update
\end{Verbatim}
\end{small}



\vspace{0.5cm} 
\item[Screen output:]
\begin{scriptsize}
\begin{verbatim}
...
__________________________________________________________________________________________________________
Add Atoms
...Size of the box to be filled.....................:     10.0000        10.0000        10.0000
...Angles of the box to be filled...................:     90.0000        90.0000        90.0000
...Origin of the box to be filled...................:      0.0000         0.0000         0.0000
...New moleculed from file..........................: h2o.pdb
...Number of molcules to add........................:          33
...Minimum distance between the molecules...........:      2.0000
__________________________________________________________________________________________________________
...
\end{verbatim}
\end{scriptsize}

\end{itemize}

\hdashrule{\textwidth}{1pt}{}
