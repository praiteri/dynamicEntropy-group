\subsubsection{Radial Pair Distribution Function - g(r)}
\index[action]{Radial Pair Distribution Function}
\index[command]{--gofr}

This action computes the Radial Pair Distribution function 
between species $\alpha$ and $\beta$, $g_{\alpha\beta}(r)$, which is defined as
\[
g_{\alpha\beta}(r) = \frac{1}{4\pi r^2 N_\alpha\rho_\beta}\sum_{i\in\alpha}\sum_{j\in\beta}\langle \delta(r_{ij}-r)\rangle
\]
where $r_{ij}$ is the distance between atoms $i$ and $j$ computed with the appropriate PBC, $\langle\rangle$ indicates the average over all the selected frames, $N_\alpha$ is the number of central atoms, and $\rho_\beta=N_\beta/V$ is the number density of the neighbouring atom.

This action also compute the Kirkwood-Buff integral as:
\[
G_{\alpha\beta}(r) = \int_0^r 4\pi r^2[g_{\alpha\beta}(r)-1]\mathrm{d}r.
\]

\hdashrule[0.5ex][x]{\textwidth}{1pt}{3mm}

\begin{itemize}
\item[Command:] \emph{--gofr}

\vspace{0.5cm}
\item[List of flags:] 
\emph{+out [gofr.out]}\\
  defines the output file name

\emph{+s [species1] [species2]} or \emph{+i [indices1] [indices2]}\\
  select the species to be used by the action.
  The species selections are done by providing a comma separated list of atom names.
  The index selections can be done by providing a comma separated list of indices or in loop format B:E:(S), where the stride is optional.
  The \emph{+mol} flag requires \emph{--top} flag, and the molecules are seleceted using the names given to them by the \emph{--top} command.

\emph{+r [6\AA]}\\
  sets the cutoff distance for the calculation of the radial pair distribution function.

\emph{+nbin [100]}\\
  defines the number of bins used for the calculation of the radial pair distribution function.

\emph{+solute}\\
  computes the g(r) for the selected atoms without using the neighbours' list.
  It is a more efficient algorithm for computing the g(r) of a solvent around a few solute atoms.

\vspace{0.5cm}  
\item[Examples:]
\begin{small}
\begin{Verbatim}[frame=single]
gpta.x --i coord.pdb traj.dcd --gofr +s Ca O1,O2 +out gofr.out +nbin 100
\end{Verbatim}
\end{small}

\vspace{0.5cm} 
\item[Screen output:]
\begin{scriptsize}
\begin{verbatim}
...
_________________________________________________________________________________________________________
Settings for the radial pair distribution function
...Cutoff distance.................................:      6.0000
...Number of bins..................................:         100
...Output file.....................................: gofr.out
...Atoms selection for --gofr
......Selection command............................:
......Atoms selected in the first group............:           1
......Atoms selected in the second group...........:          99
_________________________________________________________________________________________________________
...
\end{verbatim}
\end{scriptsize}

\vspace{0.5cm}
\item[File output:]
\begin{scriptsize}
\begin{verbatim}
# Distance | g(r) | Coordination Number | KB integral
   0.09000    0.00000    0.00000   -0.00724
   0.15000    0.00000    0.00000   -0.02443
...
\end{verbatim}
\end{scriptsize}

\end{itemize}

\hdashrule{\textwidth}{1pt}{}
