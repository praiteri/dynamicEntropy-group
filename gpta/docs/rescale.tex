\subsubsection{Rescale the Simulation Cell}
\index[action]{Rescale the Simulation Cell}
\index[command]{--rescale}

This action deforms the cell to a new shape and rescales the internal coordinates.
At the moment the system topology is ignored, which means that the molecules will be stretched after the rescaling.

\hdashrule[0.5ex][x]{\textwidth}{1pt}{3mm}

\begin{itemize}
\item[Command:] \emph{--rescale}

\vspace{0.5cm}
\item[List of flags:]

\emph{+cell [cell]}\\
  defines the size and shape of the region of space to be filled by the new molecules.
  If no cell was present, this will be the new global cell.
  See the reading coordinates action (\emph{--i}) for how the cell can be specified.

\emph{+nopos}\\
  this option scales the only the cell leaving the coordinates unchanged.

\emph{+scale [s1,s2,s3]}\\
  the lenght of the lattice vectors are multiplied by the three scaling factors [$s1,s2,s3$], but the angles are kept constant.

\vspace{0.5cm}  
\item[Examples:]
\begin{small}
\begin{Verbatim}[frame=single]
gpta.x --i coord.pdb --rescale +cell 10
gpta.x --i coord.pdb --rescale +cell 10,10,20
gpta.x --i coord.pdb --rescale +cell 10,10,10,90,90,90
gpta.x --i coord.pdb --rescale +cell 10,0,0,0,10,0,0,0,10
gpta.x --i coord.pdb --rescale +scale 1,1,3 +nopos
\end{Verbatim}
\end{small}

\vspace{0.5cm} 
\item[Screen output:]
\begin{scriptsize}
\begin{verbatim}
...
...
\end{verbatim}
\end{scriptsize}

\end{itemize}

\hdashrule{\textwidth}{1pt}{}
