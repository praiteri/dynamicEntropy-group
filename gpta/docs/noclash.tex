\subsubsection{Remove Overlapping Molecules}
\index[action]{Remove Overlapping Molecules}
\index[command]{--noclash}

This action removes overlaps between molecules by deleting the molecule that comes second in the coordinates files.
If the topology is computed before this action the entire molecule is removed when any of its atoms overlaps with another molecule.

\hdashrule[0.5ex][x]{\textwidth}{1pt}{3mm}

\begin{itemize}
\item[Command:] \emph{--noclash}

\vspace{0.5cm}
\item[List of flags:] 
\emph{+rmin [0.5]} 
  sets the minimum distance to define when atoms are considered to be overlapping.

\emph{+dry}\\
  performs a dry run to give information about how many molecules would be considered overlapping with different values for \emph{+rmin}.
  The values of \emph{+rmin} that are tested are: 0.1\AA, 0.5\AA\ and 1\AA, plus the value chosen with the \emph{+rmin} flag.

\emph{+s [species]} or \emph{+i [indices]} or \emph{+mol [molecule name]}\\
  select the species to be used in the calculation of the centre of mass using the atoms' names or their indices or the name of the molecules they belong to.
  The species selection is done by providing a comma separated list of atom names.
  The index selection can be done by providing a comma separated list of indices or in loop format B:E:(S), where the stride is optional.
  The \emph{+mol} flag requires \emph{--top} flag, and the molecules are seleceted by the names given to them by the \emph{--top} command.
  
\vspace{0.5cm}  
\item[Examples:]
\begin{small}
\begin{Verbatim}[frame=single]
gpta.x --i coord.pdb --noclash +rmin 1.1
gpta.x --i coord.pdb --top --noclash +rmin 1.1
gpta.x --i coord.pdb --top --noclash +rmin 1.1 +s K,Br
gpta.x --i coord.pdb --top --noclash +rmin 1.1 +dry
\end{Verbatim}
\end{small}

%\vspace{0.5cm} 
%\item[Screen output:]
%\begin{scriptsize}
%\begin{verbatim}
%...
%...
%\end{verbatim}
%\end{scriptsize}

\end{itemize}

\hdashrule{\textwidth}{1pt}{}
