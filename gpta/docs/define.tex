\subsubsection{Number of openMP Threads}
\index[action]{Number of openMP Threads}
\index[command]{--nt}

This command sets the number of OMP threads to be used in the calculation of the neighbours' list.
By default all available threads are used, up to a maximum of 8; unless the MPI version of the code is used, in which case only one openMP thread is used.

\hdashrule[0.5ex][x]{\textwidth}{1pt}{3mm}

\begin{itemize}
\item[Command:] \emph{--nt [number of threads]}

\vspace{0.5cm}  
\item[Examples:]
\begin{small}
\begin{Verbatim}[frame=single]
gpta.x --nt 4 --i coord.pdb
mpirun -np 8 gpta_mpi.x --nt 4 --i coord.pdb
\end{Verbatim}
\end{small}
\end{itemize}

\hdashrule{\textwidth}{1pt}{}

%%%%%%%%%%%%%%%%%%%%%%%%%%%%%%%%%%%%%%%%%%%%%%%%%%%%%%%%%%%%%%%%%%%%%%%%%%%%%%%%%%%%%%%%%%%%

\subsubsection{Screen Output}
\index[action]{Screen Output}
\index[command]{--log}

This command redirects the standard output to a file, useful for batch processes in script files.

\hdashrule[0.5ex][x]{\textwidth}{1pt}{3mm}

\begin{itemize}
\item[Command:] \emph{--log [logfile]} 

\vspace{0.5cm}  
\item[Examples:]
\begin{small}
\begin{Verbatim}[frame=single]
gpta.x --i coord.pdb --log gpta.log
\end{Verbatim}
\end{small}
\end{itemize}

\hdashrule{\textwidth}{1pt}{}

%%%%%%%%%%%%%%%%%%%%%%%%%%%%%%%%%%%%%%%%%%%%%%%%%%%%%%%%%%%%%%%%%%%%%%%%%%%%%%%%%%%%%%%%%%%%

\subsubsection{Parameters Definition}
\index[action]{Parameters Definition}
\index[command]{--define}

This command can be used to defined the value of some of the program variables.

\hdashrule[0.5ex][x]{\textwidth}{1pt}{3mm}

\begin{itemize}
\item[Command:] \emph{--define [variable] [value]}\\
  This is the list of variables that currently can be set:
  \begin{itemize}[leftmargin=2cm]
    \item[\bf{seed}] seed to initialise the random number generator
    \item[\bf{rscale}] scaling factor to define the maximum bond length
    \item[\bf{safedist}] distances wil be computed with a loop over the 27 neighbour cells
    \item[\bf{verlet}] forces the Verlet algorithm to be used in the calulation of the distances between atoms
    \item[\bf{rcov}] covalent radius of an atom for the topology calculation
    \item[\bf{bond} X,Y=2] maximum bond length between X and Y for the topology calculation
  \end{itemize}

\vspace{0.5cm}  
\item[Examples:]
\begin{small}
\begin{Verbatim}[frame=single]
gpta.x --define seed 123
gpta.x --define rscale 1.24
gpta.x --define safedist
gpta.x --define verlet
gpta.x --define rcov Ca=1.9 rcov Na=0.0
gpta.x --define bond X4,Y4=2. --i coord.pdb --top
\end{Verbatim}
\end{small}
\end{itemize}

\hdashrule{\textwidth}{1pt}{}

%%%%%%%%%%%%%%%%%%%%%%%%%%%%%%%%%%%%%%%%%%%%%%%%%%%%%%%%%%%%%%%%%%%%%%%%%%%%%%%%%%%%%%%%%%%%
