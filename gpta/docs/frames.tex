\subsubsection{Custom Selection of Frames}
\index[action]{Custom Selection of Frames}
\index[command]{--frames}

When reading a multi-frame trajectory it is possible to select a subset of configurations for processing.
This could be done via a list of comma separated list of individual frames or via a loop-type definition, B:(E):(S).
By default the end frame and stride are set to the largest single precision integer and to 1, respectively, and they can be omitted.
If only one frame is required it is possible to use the command \emph{--frame N}.

\hdashrule[0.5ex][x]{\textwidth}{1pt}{3mm}

\begin{itemize}
\item[Command:] \emph{--frames [indices]} \\

\vspace{0.5cm}  
\item[Examples:]
\begin{small}
\begin{Verbatim}[frame=single]
gpta.x --i coord.pdb traj.dcd --frames 10
gpta.x --i coord.pdb traj.dcd --frames 10:100
gpta.x --i coord.pdb traj.dcd --frames 10:100:5
gpta.x --i coord.pdb traj.dcd --frames 1,2
gpta.x --i coord.pdb traj.dcd --frame 123
\end{Verbatim}
\end{small}

\vspace{0.5cm} 
\item[Screen output:]
\begin{scriptsize}
\begin{verbatim}
...
_________________________________________________________________________________________________________
First frame to process.............................:          10
Last frame to process..............................:         100
Stride for processing frames.......................:           5
_________________________________________________________________________________________________________
...
\end{verbatim}
\end{scriptsize}

\end{itemize}

\hdashrule{\textwidth}{1pt}{}
