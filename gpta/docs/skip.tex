\subsubsection{Skiping Frames}
\index[action]{Skiping Frames}
\index[command]{--skip}

Alternative to the selection of frames with the \emph{--frames} command it is possible to just specify the stride at which the frames have to be processed.
The first frame is always processed.
This is mostly a legacy command and it will probably be removed in the future.

\hdashrule[0.5ex][x]{\textwidth}{1pt}{3mm}

\begin{itemize}
\item[Command:] \emph{--skip [N]} \\

\vspace{0.5cm}  
\item[Examples:]
\begin{small}
\begin{Verbatim}[frame=single]
gpta.x --i coord.pdb traj.dcd --skip 7
\end{Verbatim}
\end{small}

\vspace{0.5cm} 
\item[Screen output:]
\begin{scriptsize}
\begin{Verbatim}
...
_________________________________________________________________________________________________________
First frame to process.............................:           1
Last frame to process..............................:  2147483647
Stride for processing frames.......................:           7
_________________________________________________________________________________________________________
...
\end{Verbatim}
\end{scriptsize}

\end{itemize}

\hdashrule{\textwidth}{1pt}{}
