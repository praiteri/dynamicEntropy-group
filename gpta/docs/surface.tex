\subsubsection{Create Surface}
\index[action]{Create Surface}
\index[command]{--surface}

This action creates a new system unit cell orientated with the chosen Miller index parallel to the $z$ axis.
The output file contains all the possible surface cuts across the centre of mass of each molecule present in the cell.
The structures are all still 3D periodic and give the same X-ray (\emph{--xray}) scattering.
Depending on the Miller indices that are chosen the output structure will be a supercell of the initial system.
Note that the molecules must not be broken across the periodic boundary conditions for this action to work properly.
If the atoms' chrges are present the cell dipole normal to the surface is also provided.

\hdashrule[0.5ex][x]{\textwidth}{1pt}{3mm}

\begin{itemize}
\item[Command:] \emph{--surface}

\vspace{0.5cm}
\item[Prerequisite:] \emph{--top}

\vspace{0.5cm}
\item[List of flags:] 
\emph{+out [surface.pdb]}\\
  defines the output file name.

\emph{+hkl [indices]}\\
  selects the Miller indices of the surface to be create.

\vspace{0.5cm}  
\item[Examples:]
\begin{small}
\begin{Verbatim}[frame=single]
gpta.x --i coord.pdb --surface +hkl 1,0,4
\end{Verbatim}
\end{small}

\vspace{0.5cm} 
\item[Screen output:]
\begin{scriptsize}
\begin{verbatim}
...
_________________________________________________________________________________________________________
Create Surface
...Miller indices..................................: 1 0 4
_________________________________________________________________________________________________________
...
\end{verbatim}
\end{scriptsize}

\end{itemize}

\hdashrule{\textwidth}{1pt}{}
