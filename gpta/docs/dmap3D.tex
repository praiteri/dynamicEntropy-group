\subsubsection{Density Map (3D)}
\index[action]{Density Map (3D)}
\index[command]{--dmap3D}

This action computes the 3D density map for the selected atoms in a portion of the system.
The output is currently only in ``cube'' format, which can be directly read by VMD.
The units in the cube files are bohr for the definition of the grid and atoms per angstrom cube for the density.

\hdashrule[0.5ex][x]{\textwidth}{1pt}{3mm}

\begin{itemize}
\item[Command:] \emph{--dmap3D}

\vspace{0.5cm}
\item[List of flags:] 
\emph{+out [dmap3D.cube]}\\
  defines the output file name.

\emph{+s [species]} or \emph{+i [indices]} or \emph{+mol [molecule name]}\\
  select the species to be used in the calculation using the atoms' names or their indices or the name of the molecules they belong to.
  The species selection is done by providing a comma separated list of atom names.
  The index selection can be done by providing a comma separated list of indices or in loop format B:E:(S), where the stride is optional.
  The \emph{+mol} flag requires \emph{--top} flag, and the molecules are seleceted using the names given to them by the \emph{--top} command.

\emph{+nbin [50,50,50]}\\
  defines the number of bins to be used for the 3D map.

\emph{+cell [metric matrix]}\\
  defines the portion of the system where to compute the 3D density map.
  It can be defined like the system cell using 1, 3, 6 or 9 number for a cubic, orthorhombic or triclinic cell.
  The triclinic cell can be defined by either 6 numbers ($|a|,|b|,|c|,\alpha,\beta,\gamma$) or by the full metric matrix.

\emph{+origin [0.,0.,0.]}\\
  defines the origin of the subsystem.

\vspace{0.5cm}  
\item[Examples:]
\begin{small}
\begin{Verbatim}[frame=single]
gpta.x --i coord.pdb --dmap3D +out dmap3D.cube +s O2 +origin 12,34,56 +cell 10
gpta.x --i coord.pdb --dmap3D +out dmap3D.cube +s O2 +origin 12,34,56 +cell 10,10,2
gpta.x --i coord.pdb --dmap3D +out dmap3D.cube +s O2 +cell 10,10,10,90,90,90
gpta.x --i coord.pdb --dmap3D +out dmap3D.cube +s O2 +cell 10,0,0,0,10,0,0,0,10
\end{Verbatim}
\end{small}

\vspace{0.5cm} 
\item[Screen output:]
\begin{scriptsize}
\begin{verbatim}
...
_________________________________________________________________________________________________________
Compute 3D density map
...Number of bins..................................: 50 50 50
...Origin..........................................:     12.0000        34.0000        56.0000
...Region size A...................................:     10.0000         0.0000         0.0000
...Region size B...................................:      0.0000        10.0000         0.0000
...Region size C...................................:      0.0000         0.0000        10.0000
...Output file.....................................: dmap3D.cube
...Atoms selection for --dmap3D
......Selection command............................: +s O2
......Atoms selected in the first group............:        3137
_________________________________________________________________________________________________________
...
\end{verbatim}
\end{scriptsize}

\vspace{0.5cm}
\item[File output:]
\begin{scriptsize}
\begin{verbatim}
CUBE file generate by GPTA
Density reported in atoms/angstom^3
     8      0.00000      0.00000     81.25822
    50      1.74308      0.00000      0.00000
    50      0.58265      1.69986      0.00000
    50      0.00000      0.00000      1.13384
     1      1.00000      0.00000      0.00000     81.25822
...
  0.38127E-01  0.11037E+00  0.00000E+00  0.51171E-01  0.71238E-01  0.29097E-01
...
\end{verbatim}
\end{scriptsize}

\end{itemize}

\hdashrule{\textwidth}{1pt}{}
