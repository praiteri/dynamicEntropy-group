\subsubsection{Mean Square Displacement}
\index[action]{Mean Square Displacement}
\index[command]{--msd}

This action computes the Mean Square Displacement (MSD) of the selected species.
The MSD after a certain time interval $t$ is computed as
\[
MSD(t) = \frac{1}{N(t)}\sum_{j=1}^{N_{t}}\sum_{i=1}^{N_{at}}[x_i(t_0+t_j)-x_i(t_0)]
\]
where $i$ runs over the selected atoms ($N_{at}$), $t_0$ indicates the reference frame, which is reset every $N_t$ frames and $t_j$ are all the frames between two successive resets of the reference structure.
$N(t)$ is the number of configurations that have been found at a time $t=t_j-t_0$ from any given reference frame.
The periodic reset of the reference frame helps improve the statistics when computing the MSD for dilute solutes, \emph{i.e.} one ion in water.
The flag \emph{+nt} allows for controlling how often the reference frame is reset.

The mean square displacement is then used to compute the self-diffusion coefficient of the selected species.
The last 75\% of the MDS is fitted with e straight line and the slope is converted to the self-diffusion coefficient, $D$,
 using the standard Einstein relations
\[
MSD(t) = 6Dt.
\]
Note that for the calculation of self-diffusion coefficient to be correct the frames must be uniformly separated.
The time interval between the frames is set by default to 1~ps, but this can be changed using the \emph{+dt} flag.
It is important to plot the MSD from the output file to ensure that it is indeed linear to ensure that the computed self-diffusion coefficient is converged.

It is important to note that the results obtained from this action with the parallel version of GPTA depend on the number of CPUs used.
This is because each CPU computes the MSD up to $N_t$ frames, counted on the frames it receives.
Hence, because each CPU processes every N$^\mathrm{th}$ frame, where $N$ is the number of CPUs devoted to processing, the maximum time separation between the frames is
$N\times N_t\times\delta t$ picoseconds.
It is however possible to obtain the same result regardless of the number of CPU used if $N\times N_t$ is kept constant.
Please keep in mind that when more than 5 MPI processes are used, one is devoted to only reading the trajectory.

\hdashrule[0.5ex][x]{\textwidth}{1pt}{3mm}

\begin{itemize}
\item[Command:] \emph{--msd}

\vspace{0.5cm}
\item[List of flags:] 
\emph{+out [msd.out]}\\
  defines the output file name

\emph{+s [species]} or \emph{+i [indices]} or \emph{+mol [molecule names]}\\
  select the species to be used by the action.
  The species selections are done by providing a comma separated list of atom names.
  The index selections can be done by providing a comma separated list of indices or in loop format B:E:(S), where the stride is optional.
  The \emph{+mol} flag requires \emph{--top} flag, and the molecules are seleceted using the names given to them by the \emph{--top} command.

\emph{+nt [100]}\\
  maximum time separation in number of frames, $N_t$, from the reference structure to compute the MSD

\emph{+dt [1.0]}\\
  time separation between the frames in ps, which is used to compute the Self Diffusion Coefficient

\vspace{0.5cm}  
\item[Examples:]
\begin{small}
\begin{Verbatim}[frame=single]
gpta.x --i coord.pdb traj.dcd --msd +s Ca +out msd.out
gpta.x --i coord.pdb traj.dcd --msd +s O1,O2 +nt 1000
gpta.x --i coord.pdb traj.dcd --msd +mol M2 +dt 0.5
gpta.x --i coord.pdb traj.dcd --msd +i 1:300:3
\end{Verbatim}
\end{small}

\vspace{0.5cm} 
\item[Screen output:]
\begin{scriptsize}
\begin{verbatim}
...
_________________________________________________________________________________________________________
Computing mean square displacement
...Maximum number of frames................................:        1000
...Output file.............................................: msd.out
...Atoms selection for --msd
......Selection command....................................:
......Atoms selected in the first group....................:           1
_________________________________________________________________________________________________________
Self diffusion coefficient calculation
  Average D0 [10^-5 cm^2/s]................................:      0.0734
  Average correlation factor...............................:      0.9444
_________________________________________________________________________________________________________
...
\end{verbatim}
\end{scriptsize}

\vspace{0.5cm}
\item[File output:]
\begin{scriptsize}
\begin{verbatim}
# Time [ps] | MSD [A^3]
  1.000      0.90433
  2.000      2.47421
...
\end{verbatim}
\end{scriptsize}

\end{itemize}

\hdashrule{\textwidth}{1pt}{}
