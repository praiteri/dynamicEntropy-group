\subsubsection{X-ray Powder Spectrum}
\index[action]{X-ray Powder Spectrum}
\index[command]{--xray}

This action computes the X-ray powder diffraction spectrum for the input structure, which must contain the cell definition.
The output file contains the list of reflections and the powder diffraction spectrum.

The scattering intensity was computed using the traditional formula
\[
I_{hkl} \propto F_{hkl}^2 \Theta_{hkl}
\]
where 
\[
F_{hkl}^2 = \sum_i f_i \exp\bigg[-B\bigg(\frac{\sin\theta}{\lambda}\bigg)^2\bigg] \exp\bigg[2\pi i(\vec{q}\cdot\vec{x}_i)\bigg].
\]
$f_i$ are the atomic structure factors, 
$\vec{q}=(h,k,l)$ is the reciprocal space vector, 
$\vec{x}_i$ are the atomic fractional coordinates, 
$B$ is the Debye–Waller factor,
$\lambda=2d_{hkl}\sin\theta$ is the wavelength
and $d_{hkl}$ is the distance between the $hkl$ planes.

$\Theta_{hkl}$ is the Lorentz polarisation factor
\[
\Theta_{hkl} = \frac{1 + \cos^2 2\theta}{\sin^2\theta}\frac{1}{\sin\theta}
\]
where $(1 + \cos^2 2\theta)$ denotes the polarisation factor, 
$\sin 2\theta$ describes the change in irradiated volume of the crystal as a function of $2\theta$
and the last term is the powder ring distribution factor for a random distribution of crystals, \emph{i.e.} for a powder sample.
The atomic scattering factors were taken from the international tables for crystallography 
%(DOI: 10.1107/97809553602060000600) 
and have the form
\[
f = \sum_{i=1}^4 a_i \exp\bigg[-b_i \bigg(\frac{\sin\theta}{\lambda}\bigg)^2\bigg] + c.
\]
The Debey-Waller parameter $B$ was assumed to be constant, which is reasonable for harmonic vibrations,
\[
B=8\pi^2 \eta
\]
where $\eta=0.05$ for hydrogen and $\eta=0.06$ for all other elements.

\hdashrule[0.5ex][x]{\textwidth}{1pt}{3mm}

\begin{itemize}
\item[Command:] \emph{--xray}

\vspace{0.5cm}
\item[List of flags:] 
\emph{+out [xray.out]}\\
  defines the output file name.

\emph{+mini [5]}\\
  sets the minimum value of $2\theta$ for the spectrum

\emph{+max [50]}\\
  sets the maximum value of $2\theta$ for the spectrum

\emph{+d2t}\\
  sets the spectral resolution ($\delta2\theta$)

\emph{+lambda [1.540560]}\\
  sets the wavelength of the x-rays in nm

\emph{+kmax [30]}\\
  sets the maximum number of k-vectors to use in the calculation

\emph{+fwhm}\\
  sets the Full Width at Half Maximum for the thermal broadening of the peaks

\emph{+cauchy [0.5]}\\
  sets the Cauchy $n$ parameter

\vspace{0.5cm}  
\item[Examples:]
\begin{small}
\begin{Verbatim}[frame=single]
gpta.x --i coord.pdb --xray
\end{Verbatim}
\end{small}

\vspace{0.5cm} 
\item[Screen output:]
\begin{scriptsize}
\begin{verbatim}
...
_________________________________________________________________________________________________________
Computing Xray Powder Diffraction Pattern
...Output file.....................................: xray.out
...Wavelength [nm].................................:      1.5406
...Twotheta range..................................:      5.0000        50.0000
...Twotheta resolution.............................:      0.0100
...Number of kspace vectors........................:          30
...Smoothing parameters
......FWHM.........................................:      0.1000
......Cauchy n parameter...........................:      0.5000
_________________________________________________________________________________________________________
...
\end{verbatim}
\end{scriptsize}

\vspace{0.5cm}
\item[File output:]
\begin{scriptsize}
\begin{verbatim}
# index    d_hkl    twoth  intensity      (h , k , l) ...
#     1   17.062    5.175         0.       0   0  -1        0   0   1
#     2    8.531   10.361         0.       0   0   2        0   0  -2
...
# Two Theta | Intensity [a.u.]
  5.000      0.00000
...
\end{verbatim}
\end{scriptsize}

\end{itemize}

\hdashrule{\textwidth}{1pt}{}
