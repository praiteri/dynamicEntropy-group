\subsubsection{Translate the Simulation Centre of Mass}
\index[action]{Translate the Simulation Centre of Mass}
\index[command]{--fixcom}

This action can be used to traslate the centre of mass of selected atoms to a chosen position.
It is mostly useful to prepare initial configurations and for focussing the visualisation in post-processing.

\hdashrule[0.5ex][x]{\textwidth}{1pt}{3mm}

\begin{itemize}
\item[Command:] \emph{--fixcom}

\vspace{0.5cm}
\item[List of flags:] 
\emph{+s [species]} or \emph{+i [indices]} or \emph{+mol [molecule name]}\\
  select the species to be used in the calculation of the centre of mass using the atoms' names or their indices or the name of the molecules they belong to.
  The species selection is done by providing a comma separated list of atom names.
  The index selection can be done by providing a comma separated list of indices or in loop format B:E:(S), where the stride is optional.
  The \emph{+mol} flag requires \emph{--top} flag, and the molecules are seleceted using the names given to them by the \emph{--top} command.

\emph{+centre}\\
  shift the centre of mass of the selected atoms to the centre of the simulation cell

\emph{+loc [location]}\\
  shift the centre of mass of the selected atoms to the specified location

\emph{+initial}\\
  maintains the centre of mass in the same location as it was in the first frame.

\vspace{0.5cm}  
\item[Examples:]
\begin{small}
\begin{Verbatim}[frame=single]
gpta.x --i coord.pdb --fixcom +s Ca +centre
gpta.x --i coord.pdb --fixcom +mol M1 +centre
gpta.x --i coord.pdb --fixcom +s Ca +loc 1.5,2.5,3.5
gpta.x --i coord.pdb traj.dcd --fixcom +s Ca +initial
\end{Verbatim}
\end{small}

%\vspace{0.5cm} 
%\item[Screen output:]
%\begin{scriptsize}
%\begin{verbatim}
%...
%...
%\end{verbatim}
%\end{scriptsize}

\end{itemize}

\hdashrule{\textwidth}{1pt}{}
