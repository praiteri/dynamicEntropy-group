\subsubsection{Input Coordinates}
\index[action]{Input Coordinates}
\index[command]{--i}

GPTA can read the atomic coordinates in different file formats.
The coordinates' format is normally inferred from the file extension
\begin{itemize}[leftmargin=2cm]
\item[.xyz $\to$] XYZ file 
\item[.pdb $\to$] PDB file 
\item[.gin $\to$] GULP input file 
\item[.cif $\to$] Crystallographic Information File (CIF) file
\item[.dcd $\to$] CHARMM DCD file
\item[.dcd2 $\to$] CHARMM DCD file
\item[.gau $\to$] Gaussian output file
\item[.lmp $\to$] LAMMPS coordinates file
\item[.lmptrj $\to$] LAMMPS custom trajectory file
\item[.gro $\to$] GROMACS GRO file
\item[.xtc $\to$] GROMACS XTC file 
\item[.trr $\to$] GROMACS XTC file 
\end{itemize}
but it can be overridden by appending the canonical file extension (listed above) to the action flag, like in the last example below.
For the DCD, XTC and XTC binary formats that do not contain the atoms' label it is mandatory to read first an XYZ or a PDB file.
Multiple files can be specified sequentially or the command can be repeated multiple times.
Currenlty the flags must follow the file names, and they are globally applied even if more than one \emph{--i} command are used.

There are two routines that can read DCD files, by default a faster implementation is used, which however may fail if the trajectory file is corrupted.
An alternative routine can be used, where careful checks on the integrity of the files are made, can be selected with the type ``dcd2''.
This routine was taken from the libdcdfort library written by James W. Barnett (https://github.com/wesbarnett/libdcdfort).

Currently, GPTA reads only one structure per file in the ``cif'' and ``gin'' formats.

The ``xyz'' file type supports also the "extended XYZ format" used by the Atomic Simulation Environment (https://wiki.fysik.dtu.dk/ase).
For example the simulation box and the atomic charges can be included as
\begin{small}
\begin{Verbatim}[frame=single]
800
Lattice="5.44 0.0 0.0 0.0 5.44 0.0 0.0 0.0 5.44" Properties=species:S:1:pos:R:3:charge:R:1 
Na        0.00000000      0.00000000      0.00000000   1.0
Cl        1.36000000      1.36000000      1.36000000  -1.0
...
\end{Verbatim}
\end{small}

\hdashrule[0.5ex][x]{\textwidth}{1pt}{3mm}

\begin{itemize}
\item[Command:] \emph{--i [filename1] [filename2] \dots}

\vspace{0.5cm}
\item[List of flags:] 
\emph{+cell [simulation cell]}\\
  defines the shape of the simulation cell; 1, 3, 6 or 9 numbers are expected
\begin{itemize}[leftmargin=1.5cm]
  \item[1 :] cubic box
  \item[3 :] orthorhombic box defined as $|a|\ |b|\ |c|$
  \item[6 :] triclinic box defined as $|a|\ |b|\ |c|\ \alpha\ \beta\ \gamma$
  \item[9 :] triclinic box defined from the cell matrix $\vec{a}\ \vec{b}\ \vec{c}$
\end{itemize}
  The cell definition read from the coordinates file, if present, will be overridden.

\emph{+nm}\\
  assumes the input coordinates are in nm.
  This flag should not be used for coordinates files that are by definition in nm, \emph{i.e.} \verb|.gro| and \verb|.xtc|, otherwise the \AA\ to nm conversion factor will be applied twice.

\emph{+bohr}\\
  assumes the input coordinates are in bohr.


\vspace{0.5cm}  
\item[Examples:]
\begin{small}
\begin{Verbatim}[frame=single]
gpta.x --i coord.pdb 
gpta.x --i coord.pdb traj.dcd
gpta.x --i coord.pdb --i traj.dcd
gpta.x --i coord.pdb +cell 14.
gpta.x --i coord.pdb +cell 14.0,15.0,16.0
gpta.x --i coord.pdb +cell 14.0,15.0,16,90.0,90.0,120.0
gpta.x --i coord.pdb +cell 14.0,0.0,0.0,,0.0,15.0,0.0,,0.0,0.0,16.0
gpta.x --ipdb coordinates.xyz 
\end{Verbatim}
\end{small}

\vspace{0.5cm} 
\item[Screen output:]
\begin{scriptsize}
\begin{verbatim}
...
_________________________________________________________________________________________________________
Opening input coordinates files
...pdb file........................................: coord.pdb
...dcd file........................................: traj.dcd
_________________________________________________________________________________________________________
Initial cell
...Cell vector A...................................:     14.3000         0.0000         0.0000
...Cell vector B...................................:      0.0000        14.3000         0.0000
...Cell vector C...................................:      0.0000         0.0000        14.3000
...Cell lengths....................................:     14.3000        14.3000        14.3000
...Cell angles.....................................:     90.0000        90.0000        90.0000
...Volume..........................................:   2924.2070
_________________________________________________________________________________________________________
System composition
...Number of atoms found...........................:         298
......Number of unique species Ca..................:           1
......Number of unique species H...................:         198
......Number of unique species O...................:          99
------------------------------------------------------------------------
...Total mass (g/mole).............................:   1824.0600
...Density (g/cm^3)................................:      1.0358
...Total charge....................................:      0.0000
_________________________________________________________________________________________________________
Frames processing summary..........................:        3779
...Number of frames read...........................:        3779
...Number of frames processed......................:        3779
_________________________________________________________________________________________________________
...
\end{verbatim}
\end{scriptsize}

\end{itemize}

\hdashrule{\textwidth}{1pt}{}
