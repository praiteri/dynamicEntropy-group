\subsubsection{Solvent Density Map}
\index[action]{Solvent Density Map}
\index[command]{--solvation}

This action computes the 3D density map for the selected atoms around a solute molecule.
The solvent density is computed only inside the largest ellipsoid the fits inside the box around the solvent.
The output is currently only in ``cube'' format, which can be directly read by VMD.
The units in the cube files are bohr for the definition of the grid and atoms per angstrom cube for the density.

\hdashrule[0.5ex][x]{\textwidth}{1pt}{3mm}

\begin{itemize}
\item[Command:] \emph{--solvation}

\vspace{0.5cm}
\item[List of flags:] 
\emph{+id}
  defines the ID of the molecule to be use as solute.
  It has never been tested in cases where more than one solute is present.

\emph{+out [solvation.cube]}\\
  defines the output file name.

\emph{+ref [filename]}\\
  read the orientation of the solute molecule from file; the atoms in this file must be listed in the same order as they appear in the trajectory file.
  There is currently no check that this is the case.
  If this flag is not used the orientation of the solute molecule in the first frame is taken as the reference orientation.

\emph{+s [species]} or \emph{+i [indices]} or \emph{+id [molecule name]}\\
  select the species to be used in the calculation using the atoms' names or their indices or the name of the molecules they belong to.
  The species selection is done by providing a comma separated list of atom names.
  The index selection can be done by providing a comma separated list of indices or in loop format B:E:(S), where the stride is optional.
  The \emph{+id} flag requires \emph{--top} flag.

\emph{+nbin [50,50,50]}\\
  defines the number of bins to be used for the 3D map.

\emph{+cell [metric matrix]}\\
  defines the size of the 3D box around the solute molecule that is considered for the calculation of the solvent density.
  It can be defined like the system cell using 1, 3, 6 or 9 number for a cubic, orthorhombic or triclinic cell.
  The triclinic cell can be defined by either 6 numbers ($|a|,|b|,|c|,\alpha,\beta,\gamma$) or by the full metric matrix.
  Although a triclinic box can be specified, it currently works only with orthorhombic boxes and the solvent density is compute only inside an ellipsoid that fits in the box
  By default the solvent density is calculated inside a 5\AA\ sphere around the solute.

\emph{+swap [indices] ...}\\
  the positions of the selected atoms are swapped around while loooking for the optimal overlap between the current and reference molecule. 
  This is useful when looking at the solvent density around molecules that have rotating part, such as a carboxyl or a methyl group.

\emph{+smooth}\\
  enables the smoothing of the 3D map.
  At the moment it is a pretty rudimentary triangular smoothing.

\vspace{0.5cm}  
\item[Examples:]
\begin{small}
\begin{Verbatim}[frame=single]
gpta.x --i coord.pdb --top --solvation +s O2 +id M2  +cell 10,12,14  +nbin 50,50,50
gpta.x --i coord.pdb --top --solvation +id M2 +s O +cell 9 +nbin 20,20,20 +smooth +ref ace.xyz
gpta.x --i coord.pdb --top --solvation +s O2 +id M2 +cell 10  +nbin 50,50,50 +swap 1,3 8,9,10
\end{Verbatim}
\end{small}

\vspace{0.5cm} 
\item[Screen output:]
\begin{scriptsize}
\begin{verbatim}
...
_________________________________________________________________________________________________________
Computing solvation shell
...Output file.....................................: solvation.cube
...Number of bins..................................: 50 50 50
...Region size A...................................:     10.0000         0.0000         0.0000
...Region size B...................................:      0.0000        10.0000         0.0000
...Region size C...................................:      0.0000         0.0000        10.0000
...Atoms selection for --solvation
......Selection command............................: +s O2
......Atoms selected in the first group............:        3137
_________________________________________________________________________________________________________
...
\end{verbatim}
\end{scriptsize}

\vspace{0.5cm}
\item[File output:]
\begin{scriptsize}
\begin{verbatim}
CUBE file generate by GPTA
Density reported in atoms/angstom^3
     8      0.00000      0.00000     81.25822
    50      1.74308      0.00000      0.00000
    50      0.58265      1.69986      0.00000
    50      0.00000      0.00000      1.13384
     1      1.00000      0.00000      0.00000     81.25822
...
  0.38127E-01  0.11037E+00  0.00000E+00  0.51171E-01  0.71238E-01  0.29097E-01
...
\end{verbatim}
\end{scriptsize}

\end{itemize}

\hdashrule{\textwidth}{1pt}{}
