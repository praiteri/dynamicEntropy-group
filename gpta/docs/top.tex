\subsubsection{Molecular Connectivity and Topology}
\index[action]{Molecular Connectivity and Topology}
\index[command]{--top}

This action computes the atoms' connectivity and molecular topology of the system.
Although strictly speaking this action does not change the atomic coordinates, it is often a pre-requisite for most of the actions discussed in this and the following sections.
By default the topology is determined using a distance criterion for the definition of the covalent bonds.
Two atoms are considered bonded if
\[
r_{ij} < R_s(r^{cov}_i + r^{cov}_j)
\]
where $r_{ij}$ is the distance between species $i$ and $j$, $r^{cov}$ is the covalent radius of the element as defined in the \verb|elementsModule.F90| file and $R_s$ is a scaling factor set to 1.15.
The scaling factor and special bond lengths can be specified with the \emph{--define} command.

Alternatively, the molecular composition of the system can be defined by providing the list of atoms' names in each molecules.
Each molecule is defined by a comma separated list of atoms, and the atoms must be in the same order as they appear in the coordinates file.

\hdashrule[0.5ex][x]{\textwidth}{1pt}{3mm}

\begin{itemize}
\item[Command:] \emph{--top}

\vspace{0.5cm}
\item[List of flags:] 
\emph{+ion [list of species]}\\
  ensures that no covalent bond in created with the listed species.

\emph{+def [molecule1] [molecule2] \dots}\\
  with this flag the topology is not built from the neighbours list but simply using the atoms' labels.
  Each molecule has to be given as an individual argument, with the atoms belonging to each molecule provided in a comma separated list.
  All molecules in the input file(s) must have the atoms in the same order as they appear on the command line to be correctly identified.

\emph{+update}\\
  forces the topology the be updated at every frame.
  Using this flag drastically reduces the performances of the code.

\emph{+rebuild}\\
  forces the molecule that are broken across the PBC to be rebuilt.

\emph{+check}\\
  checks if any molecules are broken across the PBC and rebuilds them

\emph{+reorder}\\
  asks for the atoms in the system to be reordered and grouped by molecule.
  The program will also attempt to recognise equivalent molecule's types where the atoms are simply in different order.

\vspace{0.5cm}  
\item[Examples:]
\begin{small}
\begin{Verbatim}[frame=single]
gpta.x --i coord.pdb --top
gpta.x --i coord.pdb --top +ion Na Cl
gpta.x --i coord.pdb --top +def Ca O,H,H
gpta.x --i coord.pdb --top +update
gpta.x --i coord.pdb --top +rebuild
gpta.x --i coord.pdb --top +reorder
\end{Verbatim}
\end{small}

\vspace{0.5cm} 
\item[Screen output:]
\begin{scriptsize}
\begin{verbatim}
...
_________________________________________________________________________________________________________
Topology information
...Number of molecules found...............................:         520
...Number of unique molecules found........................:           2
....Molecule type M1
......Number of molecules..................................:           1
......Concentration [M]....................................:      0.1081
......Number of atoms in molecule..........................:           1
......Atoms types in molecule..............................: Ca
......Number of atoms in molecule..........................:           1
......Molecular charge.....................................:      0.0000
....Molecule type M2
......Number of molecules..................................:         519
......Concentration [M]....................................:     56.0923
......Number of atoms in molecule..........................:           3
......Atoms types in molecule..............................: O2,H2,H2
......Number of atoms in molecule..........................:           3
......Molecular charge.....................................:      0.0000
_________________________________________________________________________________________________________
...
\end{verbatim}
\end{scriptsize}

\end{itemize}

\hdashrule{\textwidth}{1pt}{}
