\subsubsection{Align frames}
\index[action]{Align frames}
\index[command]{--align}

This action aligns the frames to minimise the RMSD for the selected molecule.
It is analogous to the \emph{--solvation} command, but instead of computing the density profile it bdumps out the rotated coordinates. 
A selection can be made to write out the aligned positions of a subset of atoms

\hdashrule[0.5ex][x]{\textwidth}{1pt}{3mm}

\begin{itemize}
\item[Command:] \emph{--solvation}

\vspace{0.5cm}
\item[List of flags:] 
\emph{+id}
  defines the ID of the molecule to be use as solute.
  It has never been tested in cases where more than one solute is present.

\emph{+out [aligned.dcd]}\\
  defines the output file name.

\emph{+ref [filename]}\\
  read the orientation of the solute molecule from file; the atoms in this file must be listed in the same order as they appear in the trajectory file.
  There is currently no check that this is the case.
  If this flag is not used the orientation of the solute molecule in the first frame is taken as the reference orientation.

\emph{+swap [indices] ...}\\
  the positions of the selected atoms are swapped around while loooking for the optimal overlap between the current and reference molecule. 
  This is useful when looking at the solvent density around molecules that have rotating part, such as a carboxyl or a methyl group.

\emph{+s [species]} or \emph{+i [indices]} or \emph{+mol [molecule names]}\\
  select the species to be used by the action.
  The species selections are done by providing a comma separated list of atom names.
  The index selections can be done by providing a comma separated list of indices or in loop format B:E:(S), where the stride is optional.
  The \emph{+mol} flag requires \emph{--top} flag, and the molecules are seleceted using the names given to them by the \emph{--top} command.

\vspace{0.5cm}  
\item[Examples:]
\begin{small}
\begin{Verbatim}[frame=single]
gpta.x --i coord.pdb traj.dcd --top --align +id M2 +out aligned.dcd
gpta.x --i coord.pdb traj.dcd --top --align +id M2 +mol M2 +out aligned.dcd
\end{Verbatim}
\end{small}

%\vspace{0.5cm}
%\item[Screen output:]
%\begin{scriptsize}
%\begin{verbatim}
%...
%...
%\end{verbatim}
%\end{scriptsize}

\end{itemize}

\hdashrule{\textwidth}{1pt}{}
